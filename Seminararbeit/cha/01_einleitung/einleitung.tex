%-------------------------------------------
%--  Chapter							  --
%-------------------------------------------
%10% --> 1 Seite
\chapter{Einleitung}
\label{cha:einleitung}
%Die Quelle der Einleitung ist nur https://www.trwaftermarket.com/de/news/die-ganze-kraft-des-lenkens/
%
%https://karriere.mercedes-amg.com/speedletter-012017/die-lenkung/ %Bild einer Lenkung mit Riemenantrieb (Project One)
%https://www.total911.com/opinion-in-defence-of-the-porsche-991s-electric-power-steering/ %Porsche 991 EPS mit %Riemenantrieb

Gewicht, Kosten und Kohlenwasserstoffemission sollen bei allen Herstellern gesenkt werden.
Teilsysteme müssen deshalb weiter optimiert werden.
Entkoppelung der Lenkung vom Motor wenn nicht gelenkt wird.
Herkömmliche (veraltete) Lenkungen werden von einem Volumenstrom unterstützt, der von der Servopumpe gefördert wird. 
Die Pumpe wird dabei permanent vom Motor über einen Riemen betrieben. 
Das hat zur Folge, dass konstant Energie und somit Kraftstoff verbraucht wird. 
Durch die Verwendung einer rein elektrischen EPS-Lenkung (EPS-Electric Power Steering) kann die Lenkunterstützung von dem Verbrennungsmotor entkoppelt werden
entkoppelung ist für die Elektrisierung der Fahrzeuge nötig. 
Zukünftig kann die Pumpe nicht mehr von dem Verbrennungsmotor betrieben werden

EPS-Lenkung hat das aktuell höchste Kraftstoffeinsparpotenzial und ist zusätzlich Umweltbewusster als Lenkungen die von Hydrauliköl unterstützt werden, da das Umweltbelastende Öl eingespart werden kann.  
Effizienz steigt durch die Verwendung einer EPS-Lenkung 

Die EPS-Lenkungsarten werden im groben zwischen Zahnstangen- und Lenksäulen-EPS unterschieden. 
Die Lenksäulen-EPS wurde ursprünglich für das Kleinwagen-Segment entwickelt und ist inzwischen weltweit sehr verbreitet.
Die Lenksäulen-EPS hat den Vorteil, dass die Unterstützungseinheit in der Fahrgastzelle und nicht im Motorraum platziert ist.
Dadurch sinken die Anforderungen an die Lenkung in Hinsicht auf Abdichtung und Temperaturbeständigkeit drastisch im Vergleich zu der im Motorraum verbauten Zahnstangenangetriebenen Lenkung. 
   
Fahrzeuge aus dem Premium-Segment werden mit Riemenantrieb (Zahnstangenantrieb) ausgestattet. 
Dabei wird ein bürstenloser Motor mit einem Zahnriemen und einer Kugelumlaufbaugruppe mit Antriebsrad und Lager in einer Einheit direkt an der Zahnstange verbaut.
Dadurch kann eine geringere Trägheit und Reibung für Hochkraftanwendungen sowie ein direkteres Lenkgefühl und erstklassiges Ansprechverhalten erzeugt werden. 

Besonders für Fahrzeuge des deutschen D-Segments (US-MIttelkalsse) und darüber geeignet. 
Eine EPS-Lenkung mit Riemenantrieb ist in der Lager Zahnstangenkräfte bis zu \unit[18]{kN} bei \unit[12]{V} aufzubringen.
Der mechatronische Kern (Systemarchitektur, Sicherheit, Sensor, Motor, ECU und Software-Designs) der Lenkung kann sowhl für der Riemenantrieb wie auch für den Lenksäulenantrieb verwedet werden. 

Durch den gezielten Einsatz nur im benötigten Situationen, kann eine Kraftstoffeinsparung bei einem Verbrennungsmotor bis zu \unit[0,33]{l} pro \unit[100]{km} erwirkt werden. Das entspricht in etwas einer CO\textsubscript{2} Reduktion von \unit[8]{g} pro \unit[]{km}. 
