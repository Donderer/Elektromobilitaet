%-------------------------------------------
%--  Chapter								 --
%-------------------------------------------
\chapter{Zusammenfassung und Ausblick}
\label{cha:ausblick}
%10% --> 1 Seite





%Seit der Liberalisierung des Strommarktes in Deutschland und dem damit verbundenen, freien Wettbewerb unterliegt die Akteursstruktur einer stetigen Änderung. Zentrale, monopolistische Strukturen in Form von großen Energieversorgern werden im Laufe des Energiewendeprozesses nachhaltig von dezentralen, kleinen und heterogenen Akteuren verdrängt. Diese starke Dezentralisierung wirkt sich sowohl auf die Besitzverhältnisse, sowie auch auf die technologische Zusammensetzung von Erzeugungsanlagen aus (vgl.\,\autocite{Energiewende2014}).
% 
%Damit diese Umstrukturierung mit einem funktionierenden, freien und gleichzeitig zielgerichteten Wettbewerb am Energiemarkt in Einklang gebracht werden kann, müssen dafür klare Richtlinien vorhanden sein. Zu diesem Zweck wurden vor allem im Zuge der Umstellung zum Ausschreibungsverfahren Gesetzesänderungen im \acs{EEG} veranlasst und neue Leitgedanken zur Erreichung der Ziele gefasst. Als wesentlicher Punkt ist eine kosteneffiziente Sicherung des kontinuierlichen und kontrollierten Ausbaus von erneuerbaren Energien zu nennen, während dabei die Akteursvielfalt durch gleiche und faire Chancen für alle Akteure gewahrt werden soll. Zudem sollen zur Steigerung der Wettbewerbsfähigkeit \acp{EE} stärker in den Strommarkt integriert werden.
%
%Durch die Umstellung zum Ausschreibungsverfahren kommen jedoch erhöhte Transaktions- und Finanzierungskosten auf Investoren zu, welche eine erhebliche Markteintrittsbarriere für kleinere Akteure darstellen kann. Weiterhin wird die Preissteuerung durch eine Mengensteuerung ersetzt, die die bislang dynamische Entwicklung des Ausbaus von \acp{EE} massiv deckeln wird. Durch diese beiden und durch weitere Einschränkungen im Marktgeschehen werden die Chancen zur Umsetzung von Projekten im Bereich der Erneuerbaren verringert und der kontinuierliche und kontrollierte Ausbau gefährdet (vgl.\,\autocite{Ohlhorst2016}).
%
%Das Engagement von Bürgerinnen und Bürgern haben mit der Einspeisevergütung eine eminente und treibende Kraft der Energiewende dargestellt. Mit den Rahmenbedingungen  der Ausschreibung wird, obwohl der Erhalt der Akteursvielfalt ein politisch klar definiertes Ziel ist, das Ausschreibungsverfahren voraussichtlich wieder zurück zu stärker zentralisierten Besitzstrukturen der Erzeugungsanlagen führen. Es ist zu vermuten, dass die Dynamik des bisherigen Bürgerengagements reduziert wird (vgl.\,\autocite{Ohlhorst2016}).
%
%Zur Erhöhung der Akteursvielfalt erfolgt beispielsweise in Dänemark die Maßnahme, Konzessionen im Ausmaß von mindestens 20\,Prozent des Projektvolumens an lokale Akteure zu vergeben (vgl.\,\autocite{Energiewende2014}).
%
%Beispielhaft wird eine zukünftige Chancengleichheit für alle Akteure einer wissenschaftlichen Analyse zufolge mittels einem Bewertungsmodell hinsichtlich der Wirtschaftlichkeit für \acp{WKA} nicht erreicht. Ein systematischer Wettbewerbsnachteil für \acp{WKA} wird durch die Standortabhängigkeit mit einem Gütefaktor kleiner 70\,Prozent bestehen, sofern keine Gesetzesanpassungen erfolgen (vgl.\,\autocite{Wag2016}).
%
%Die Realisierungsrate von Projekten die ausgeschrieben werden, fiel in anderen Ländern erfahrungsgemäß sehr gering aus. Um dies in Deutschland zu vermeiden, sind Strafzahlungen vorgesehen. Dennoch gibt es die Möglichkeit zusätzlich finanzielle Garantien seitens der Anbieter zu verlangen. In Italien, Brasilien und Dänemark werden sogenannte Finanzierungsbonds eingesetzt. Anbieter müssen hierbei vor und/oder nach Zusage einen Teil des Projektvolumens als Sicherstellung hinterlegen (vgl.\,\autocite{Energiewende2014}). Eine stärkere Integration der \acp{EE} ist ebenso mit einer Konkurrierbarkeit mit konventionellen Anlagen verbunden. Hierzu müssen \acs{EE}-Anlagen ausreichend wirtschaftlich sein. Eine Sensitivitätsanalyse nach \autocite{Wag2016} hat ergeben, dass die Höhe der Vergütungs- sowie der Investitionskosten die größten Einflüsse auf die Wirtschaftlichkeit darstellen. Der festgelegte Höchstpreis durch die \acs{BNetzA} und die Entwicklung der Investitionskosten bestimmen letztlich die zukünftige Wettbewerbsfähigkeit.
%
%Zusammenfassend ist eine Systemänderung, d.h. eine Änderung des Förderungsverfahrens für \acs{EE}-Anlagen, zu befürworten, denn insbesondere aus Netzsicht ist ein regulierter Ausbau der \acp{EE} zur Einhaltung von Netzrestriktionen notwendig. Ein Gesamtkonzept der Energiewende zur Sicherung des stabilen Netzes auch bei einem hohen Anteil an \acp{EE} wurde noch nicht definiert. Es gilt dennoch, Gesetzesänderungen zeitnah und agil vorzunehmen, so dass die fest definierten Ziele in Zukunft gewahrt werden. 