%-------------------------------------------
%--  Chapter								 --
%-------------------------------------------
%10% --> 1 Seite
\chapter{Notwendigkeit}
\label{cha:notwendigkeit}
warum ist model based design super...
-Model-based systems engineering


Zeit und Kostenplan können vollkommen aus dem Ruder laufen.
Magisches Tool wird  gesucht um die Kontrolle zu behalten.
Die Standish Group hat 1994 den ersten CHAOS-Bericht herausgebracht.
Das war noch die Anfangsphase des Softwareentwicklung und laut dem Bericht haben sich  31,1 Prozent aller in diesem Jahr gestarteten IT-Projekte als komplette Fehlplanungen erwiesen.
Dabei wurden über 100.000 IT-Projekte in den USA untersucht.
in dem Bericht wurden die Projekte in drei Kattegorien eingeteilt. Successfull, challenged und failed. 
Rund 10 Jahre später wurde die Untersuchung wiederholt. Dabei ergab sich das in Tabelle \ref{Ergebnisse des CHAOS-Reports} dargestellte Ergebnis.

\begin{table}
	\caption{Ergebnisse des CHAOS-Reports von ...}
	\begin{center}
	\begin{tabular}{|l|c|c|l|} \hline
		Kattegorie & 1994 & 2004 & Veränderung \\ \hline
		Successful & 16\% & 34\% & Verdoppelt \\ \hline
		Challenged & 53\% & 51\% & Nahezu Konstant \\ \hline
		Failed 	   & 31\% & 15\% & Halbiert \\ \hline					
	\end{tabular}
	\end{center}
	\label{Ergebnisse des CHAOS-Reports}
\end{table} 

