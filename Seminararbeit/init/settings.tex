%-------------------------------------------
%--  Settings							  --
%-------------------------------------------

%\RequirePackage{fix-cm}							%Standardschrift - Computer Modern, dickere Schrfit in seriefenloser Schrift --> Komaskript berichtigt die Schrift bei Satzspiegelberechnung
\documentclass[
	12pt, 										%Schriftgröße
	draft=false,									%Überlänge des Textes markieren
	DIV=13,										%Unterteilung
	%BCOR=7mm	,									%Bindekorrektur
	a4paper, 									%Papierformat
	ngerman, 									%Silbentrennung und Umlaute, dt. Sprachpaket
	titlepage,									%Titelseite
	listof=totoc,								%Verzeichnisse im Inhaltsverzeichnis aufführen
	bibliography=totoc,							%Literaturverzeichnis im TOC
	headinclude=true,							%Satzspiegelberechnung
	footinclude=false,							%Satzspiegelberechnung
	headsepline=true,							%Linie Kopfzeile
	footsepline=false,							%Linie Fußzeile
	captions=tableheading,						%Platz zw. Caption und Tabelle
	abstract=false,								%Zusammenfassung vor Dokument
	parskip=true,								%Absätze statt Einrückungen
]{scrreprt}
%\recalctypearea									%Neuberechnung Satzspiegel bei DIV=calc

%-------------------------------------------
%--  Load Packages						  --
%-------------------------------------------
\usepackage[T1]{fontenc}							%Suchfkt. in PDF, Kopieren von Umlauten
\usepackage{babel}								%Sprachpakete, global in documentclass definiert
\usepackage[utf8]{inputenc}						%Umlaute, Editor UTF-Kodieriung aktivieren
\usepackage{lmodern}								%Schriftart inkl. Mathe
\usepackage[babel,german=quotes]{csquotes}		%Deutscher Stil
\usepackage[math]{blindtext}						%Blindtext mit Option z.B. Mathe [math]
\usepackage{amsmath,amssymb}						%Mathepakete
\usepackage[decimalsymbol=comma]{siunitx}		%SI-Einheiten
\usepackage{microtype}							%Optischer Randausgleich, Trennstriche in Rand
\usepackage[automark]{scrlayer-scrpage}			%Kopf- und Fußzeilenpaket (früher scrpage2)
\usepackage{biblatex}							%Literaturverzeichnis mit BIBER
\usepackage[printonlyused]{acronym}				%Abkürzungsverzeichnis, nur verwendete Abkürzungen darstellen
\usepackage{appendix}							%Anhang
\usepackage{epigraph}							%Zitat zu Beginn
\usepackage{eurosym}								%Eurosymbol
\usepackage{tabularx}							%Tabelle X kann \newline + autom. Spaltenbreite
\usepackage[table]{xcolor} 						%Tabellen farbig
\usepackage{array}								%Vertikale Ausrichtung m-Spalte, b-Spalte
\usepackage{booktabs}							%SW Linien Torule,Midrule,Bottomrule
\usepackage{multirow}							%Gemittelte Ausrichtung Tabellenzelle
\usepackage{ragged2e}							%Links-/Rechstbündig RaggedRight ~= raggedright! 
\usepackage{units}								%Einheiten in folgendem Format \unit[100]{°C}

%-------------------------------------------
%--   URL								  --
%-------------------------------------------
\usepackage{url}
\setcounter{biburlnumpenalty}{100}
\setcounter{biburlucpenalty}{100}
\setcounter{biburllcpenalty}{100}

%-------------------------------------------
%--   Images							  --
%-------------------------------------------
\usepackage{graphicx}							%Bilder einbinden
\usepackage{epstopdf}							%EPS2PDF Converter
\usepackage{subfig}								%Bildunterteilung

%-------------------------------------------
%--  Global Variables & Commandos		  --
%-------------------------------------------
\newcommand{\authorname}{Fabian Donderer}
\newcommand{\subjectname}{Seminararbeit}	
\newcommand{\titlename}{Methodische und modellbasierte Betrachtung einer EPS-Lenkung}
\newcommand{\subtitlename}{Nachhaltige modellbasierte Elektromobilität}
\newcommand{\betreuer}{Prof.\ Dr.\ Vahid Salehi\\ \  \  }
\newcommand{\publishersname}{%
	Hochschule München\protect\\
	Hochschule für angewandte Wissenschaften München\protect\\
	Fakultät für angewandte Naturwissenschaften und Mechatronik}
\newcommand{\keywordslist}{Seminararbeit, Elektromobilität, Mechatronische Systeme, Methodische Betrachtung, Modellbasierte Betrachtung, EPS-Lenkung}
\newcommand{\pdfcreatorlist}{MiKTeX, LaTeX with hyperref and KOMA-Script}

\newcommand{\img}[5]{
	\begin{figure}[htbp]\centering
		\includegraphics[width=#1\linewidth,clip,trim=0mm 0mm 0mm 0mm]{#2}
		\caption[#3]{#4}
		\label{fig:#5}
	\end{figure}}
	
\newcommand{\subimg}[8]{	
	\begin{figure}[htbp]
		\centering
		\subfloat[#1]{
			\includegraphics[width=#2\linewidth,clip,trim=0mm 0mm 0mm 0mm]{#3}}	%trim/clip li, un, re, ob
			\qquad
			\subfloat[#4]{
			\includegraphics[width=#2\linewidth,clip,trim=0mm 0mm 0mm 0mm]{#5}}	%trim/clip li, un, re, ob
			\caption[#6]{#7}
		\label{fig:#8}
	\end{figure}}
	
%-------------------------------------------
%--   Silbentrennung			   		  --
%-------------------------------------------
\hyphenation{ 
  %Tech-no-logie-ver-drän-gung
} 

%-------------------------------------------
%--   SI-Units							  --
%-------------------------------------------
\newunit{\ct}{ct}
\newunit{\GWh}{GWh}

%-------------------------------------------
%--   References						  --
%-------------------------------------------
\usepackage{varioref}							%Reihenfolge beachten für Cross-Referencing
\usepackage[
	hyperfootnotes,								%Option Rücksprung von Fußnote in Text
	bookmarksopenlevel=1,						%PDF Öffnet die Chapter in der Vorschau					
]{hyperref}				
\usepackage{cleveref}							%Referenzierung mit Seite und Bezeichner


%-------------------------------------------
%--  Meta PDF							  --
%-------------------------------------------
\hypersetup{pdfinfo={
  Title=\titlename,
  Author=\authorname,
  Subject=\subjectname,
  Keywords={\keywordslist},
  pdfcreator={\pdfcreatorlist},
}}

%-------------------------------------------
%--  Imagepath							  --
%-------------------------------------------
\graphicspath{{./img/}}							%Grafiken in einem Unterordner speichern

%-------------------------------------------
%--  Formatting							  --
%-------------------------------------------
\setkomafont{disposition}{\normalcolor\bfseries}	%Überschriften mit Serifen
%\setlength{\parindent}{0em}						%Erstzeileneinzug/Einrückung on/off

%-------------------------------------------
%--  Header & Footer					  --
%-------------------------------------------
\pagestyle{scrheadings}							%Kopf- und Fußzeile
\renewcommand*{\chapterpagestyle}{scrheadings} 	%Kopf- und Fußzeile auch auf Kapitelanfangsseiten
\renewcommand{\headfont}{\normalfont}			%Schriftform der Kopfzeile
%\setlength\footskip{110pt}						%Höhe Fußzeile

%% Header
\ihead{\headmark} 								%links, Kapitel
\chead{}										%mitte
\ohead{}										%rechts

%% Footer
\ifoot{}										%links
\cfoot{}										%mitte
\ofoot{\pagemark}								%rechts